
\documentclass{beamer}


\usepackage[utf8]{inputenc}%esto permite meter tildes sin el coso
\usepackage[spanish]{babel}
\usepackage{listings}
\usepackage{algpseudocode}

\usepackage{amsmath}
\newcommand{\quotes}[1]{``#1''}
\newcommand{\bigo}[0]{$\mathcal{O}$}
%e\newcommand{\elevar}[2]{$ {#1}^{#2} $ }

% There are many different themes available for Beamer. A comprehensive
% list with examples is given here:
% http://deic.uab.es/~iblanes/beamer_gallery/index_by_theme.html
% You can uncomment the themes below if you would like to use a different
% one:
\usetheme{Madrid}


\newtheorem{idea-imp}{¡Idea Importante!}


\title{Concurso Area Teoría}

% A subtitle is optional and this may be deleted
%\subtitle{Optional Subtitle}


\author{Brian Bokser}

% - Give the names in the same order as the appear in the paper.
% - Use the \inst{?} command only if the authors have different
%   affiliation.

%\institute[UBA] % (optional, but mostly needed)

\date{}
% - Either use conference name or its abbreviation.
% - Not really informative to the audience, more for people (including
%   yourself) who are reading the slides online

%\subject{Theoretical Computer Science}
% This is only inserted into the PDF information catalog. Can be left
% out. 

% If you have a file called "university-logo-filename.xxx", where xxx
% is a graphic format that can be processed by latex or pdflatex,
% resp., then you can add a logo as follows:

% \pgfdeclareimage[height=0.5cm]{university-logo}{university-logo-filename}
% \logo{\pgfuseimage{university-logo}}

% Delete this, if you do not want the table of contents to pop up at
% the beginning of each subsection:
\AtBeginSubsection[]
{
  \begin{frame}<beamer>{Outline}
    \tableofcontents[currentsection,currentsubsection]
  \end{frame}
}

% Let's get started
\begin{document}

\maketitle

\section{Introducci\'on}

\begin{frame}{Introducción}
    \begin{itemize}
	\item Materia :  \emph{Teoría de Lenguajes}
	\vspace{2em}
      \item Práctica : 5 - \emph{Lenguajes regulares y lema de pumping}

    \end{itemize}

\end{frame}

\begin{frame}{Contexto}
	\begin{itemize}
          \pause
        \item Los estudiantes conocen el lema de pumping de la teórica
          \vspace{2em}
          \pause
      \item Probaron que ciertos lenguajes no son regulares mediante aplicación directa del lema
        \vspace{2em}
        \pause
      \item Falta un ejemplo donde valga el lema pero el lenguaje no sea regular
        \vspace{2em}
	\end{itemize}
\end{frame}


\begin{frame}{Enunciado}
	

Dado $L = \{~0^i1^j ~| ~i>j \lor i~ par\}$

\begin{itemize}
\item[a] Demostrar que L cumple: 
$$ \forall \alpha (\alpha \in L \land |\alpha| \geq 2 \Rightarrow \exists x, y, z $$ 
$$  (\alpha = xyz \land |xy| \leq 2 \land |y| \geq 1 \land \forall k (xy^kz \in L))) $$

\item[b] Demostrar que L no es regular
\end{itemize}


\end{frame}


\begin{frame}{Motivación}
  El ejercicio nos sirve para:
  \begin{itemize}
    \pause
  \item Poner en práctica props. de lenguajes regulares
    \pause
  \item Entender el Lema de pumping
    \pause
  \item Enfatizar L reg $\implies$ Lema de pumping pero \textbf{NO} la vuelta
  \end{itemize}
\end{frame}

\begin{frame}{Parte A}
   $L = \{0^i1^j : i > j \lor i \textnormal{ par}$ \}
$$ \forall \alpha (\alpha \in L \land |\alpha| \geq 2 \Rightarrow \exists x, y, z $$ 
  $$  (\alpha = xyz \land |xy| \leq 2 \land |y| \geq 1 \land \forall k (xy^kz \in L))) $$

  \pause
  Sea $\alpha \in L$, $\alpha = 0^i1^j$ \\
  \pause
  Opciones:
  \begin{itemize}
  \pause
  \item i par
    \begin{itemize}
      \pause
      \item $i = 0$
      \pause
      \item $i \geq 2$
      \pause
    \end{itemize}
  \item $i\textnormal{ impar} \land i \geq j$
  \end{itemize}
\end{frame}

\begin{frame}{L no es regular}
  $L' = \{0^i1^j : i \leq j \land i \textnormal{ impar}$ \} \\
  \vspace{1em}
  Supongamos L regular y lleguemos a un absurdo \\
  \vspace{1em}
  \pause
  L regular $\implies L'$ regular (en pizarrón) \\
  \pause
  Por lema de pumping: \\
  \pause
  $$\exists n_0 (\forall \alpha \in L'(|\alpha| \geq n_0 \implies$$ \\
  $$\exists x, y, z (\alpha = xyz \land |xy| \leq n_0 \land  |y| > 0 \land \forall k (xy^Kz \in L)))) $$

  \pause
  Pero demostraremos la negación:
  \pause
  $$\forall n_0 \textnormal{ } (\exists \alpha \in L'\textnormal{ } (|\alpha| \geq n_0 \textnormal{ } \land  $$ \\
  $$\forall x, y, z (\alpha = xyz \land |xy| \leq n_0 \land  |y| > 0 \land \exists k (xy^Kz \notin L)))) $$

\end{frame}

\begin{frame}{Negación de pumping}
  $$\forall n_0 \textnormal{ } (\exists \alpha \in L'\textnormal{ } (|\alpha| \geq n_0 \textnormal{ } \land  $$ \\
  $$\forall x, y, z (\alpha = xyz \land |xy| \leq n_0 \land  |y| > 0 \land \exists k (xy^Kz \notin L)))) $$
  \vspace{2em}
  \begin{center}
    En pizarrón
  \end{center}
\end{frame}

\begin{frame}{¿Preguntas?}
	\begin{center}
		\includegraphics[scale=0.4]{homero-pelicula.jpg}
	\end{center}
\end{frame}


\end{document}


