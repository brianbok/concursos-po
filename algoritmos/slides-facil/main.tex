\documentclass[pdf]{beamer}
\newcommand{\quotes}[1]{``#1''}

\mode<presentation>
{
  \usetheme{Warsaw}
  \useoutertheme{infolines}
  \usecolortheme[RGB={28,13,151}]{structure}
  %\usetheme[height=7mm]{Rochester}
  % or ...

  \setbeamercovered{invisible}
  % or whatever (possibly just delete it)
}

\usepackage{ upgreek }
\usepackage{verbatim} 
\usepackage{listings}
\usepackage{tikz}
\usepackage{algpseudocode}

\usetikzlibrary{arrows}
\usetikzlibrary{shapes}
\tikzstyle{block}=[draw opacity=0.7,line width=1.4cm]

\lstloadlanguages{C++}
\lstnewenvironment{code}
	{%\lstset{	numbers=none, frame=lines, basicstyle=\small\ttfamily, }%
	 \csname lst@SetFirstLabel\endcsname}
	{\csname lst@SaveFirstLabel\endcsname}
\lstset{% general command to set parameter(s)
	language=C++, basicstyle=\footnotesize\ttfamily, keywordstyle=\slshape,
	emph=[1]{tipo,usa}, emphstyle={[1]\sffamily\bfseries},
	morekeywords={tint,forn,forsn},
	basewidth={0.47em,0.40em},
	columns=fixed, fontadjust, resetmargins, xrightmargin=5pt, xleftmargin=15pt,
	flexiblecolumns=false, tabsize=2, breaklines,	breakatwhitespace=false, extendedchars=true,
	numbers=left, numberstyle=\tiny, stepnumber=1, numbersep=9pt,
	frame=l, framesep=3pt,
}
\usepackage{multicol}

\usepackage[spanish]{babel}
% or whatever

\usepackage[utf8]{inputenc}
% or whatever

\usepackage{times}
\usepackage[T1]{fontenc}
% Or whatever. Note that the encoding and the font should match. If T1
% does not look nice, try deleting the line with the fontenc.
\usepackage{enumitem}
\usepackage{booktabs}

\newtheorem{enun}{Enunciado}
\newtheorem{idea}{Idea importante}


\title[Prueba de oposición] % (optional, use only with long paper titles)
{Prueba de oposición - Algoritmos 2016}

\author[Brian Bokser] % (optional, use only with lots of authors)
{Brian Bokser}
% - Give the names in the same order as the appear in the paper.
% - Use the \inst{?} command only if the authors have different
%   affiliation.
\institute[UBA-FCEN] % (optional, but mostly needed)
{
  Facultad de Ciencias Exactas y Naturales\\
  Universidad de Buenos Aires
}


% Ac� se puede insertar el logo de la UBA
% \pgfdeclareimage[height=0.5cm]{university-logo}{university-logo-filename}
% \logo{\pgfuseimage{university-logo}}


% Delete this, if you do not want the table of contents to pop up at
% the beginning of each subsection:
\AtBeginSubsection[]
{

  \begin{frame}<beamer>{Contenidos}
    \tableofcontents[currentsection,currentsubsection]
  \end{frame}
	
}


% If you wish to uncover everything in a step-wise fashion, uncomment
% the following command: 

%\beamerdefaultoverlayspecification{<+->}


\begin{document}
\pgfdeclarelayer{background}
\pgfsetlayers{background,main}
\begin{frame}
  \titlepage
\end{frame}

\begin{frame}{Contenidos}
  \tableofcontents
  % You might wish to add the option [pausesections]
\end{frame}



\section{Introducci\'on}

\begin{frame}{Introducci\'on}
    \begin{itemize}
	\item Materia :  \emph{Algoritmos y Estructuras de Datos II}
	\vspace{2em}
	\item Práctica : \emph{Séptima práctica - Dividir y Conquistar}

    \end{itemize}

\end{frame}

\begin{frame}{Contexto}
    \par{Los alumnos ya tuvieron la teórica de Divide and Conquer. }
    \vspace{2em}
	
    \par{Conocen las ideas clásicas, ahora tienen que ejercitarlas.}
    \vspace{2em}
	
    \par{Conocen el teorema maestro, y pueden usar alguno de sus casos para determinar 
    la complejidad de una recursión.}
\end{frame}

\section{Prácticas de la materia} 
\frame{\frametitle{Presentación} 
\begin{itemize}
    \item Primera parte
    \begin{itemize}
        \item Especificación con Tipos Abstractos de Datos
        \item Demostración de propiedades
        \item Diseño: invariante de representación y función de abstracción
    \end{itemize}
    \item Segunda Parte
    \begin{itemize}
        \item Complejidad Algorítmica
        \item Diseño: elección de estructuras de datos 
        \item Ordenamiento
        \item \textbf{Dividir y Conquistar}
    \end{itemize}
\end{itemize}
}

\begin{frame}{Enunciado}

\begin{enun}
    Suponga que se tiene un método potencia que, dada un matriz cuadrada A de orden 4 $\times$ 4 
    y un número n, computa la matriz $A^n$ . 
    Dada una matriz cuadrada A de orden 4 $\times$ 4 y un número natural n que es potencia de 2 
    (i.e., n = 2 $\times$ k para algun k $\geq$ 1)
    , desarrollar, utilizando la técnica de dividir y conquistar y el método potencia,
un algoritmo que permita calcular: 

$A^1$ + $A^2$ + . . . + $A^n$

Calcule el número de veces que el algoritmo propuesto aplica el método potencia. Si no es estrictamente menor que O(n), 
resuelva el ejercicio nuevamente.
\end{enun}

\end{frame}

\begin{frame}{Motivación}
    
\end{frame}

\begin{frame}{¡Al ataque!}
    \par{Primer idea: \pause Buscar la fórmula de serie geométrica. \pause No parece servir para \alert{matrices}}
    \pause
    \vspace{1em}
    \par{Segunda idea: \pause algoritmo ingenuo. \pause \alert{O($n$)} llamados a potencia.}
    \pause
    \vspace{1em}
    \par{Buscamos algo sublineal...}
    \pause
    
    \vspace{1em}
    
    ¿Como resolver un problema de Divide and Conquer?
    
    \pause
    
    \begin{idea}
	¡Queremos algun subproblema que nos sirva para dar una solución al problema original!
    \end{idea}
\end{frame}

\begin{frame}{Buscando el subproblema ganador}
    \par{Queremos encontrar alguna propiedad que nos ayude a dar esta formulación en base a subproblemas.}
    \vspace{1em}
    
    \pause
    
    \par{Idea: }
    
    ($A^1$ + $A^2$ + . . . + $A\textsuperscript{n/2}$) $\times$ $A^\textsuperscript{n/2}$ = ($A^\textsuperscript{n/2 + 1}$ + $A^\textsuperscript{n/2 + 2}$ + . . . + $A^n$)
    
    \pause
    \vspace{1em}
    
    \par{$A^1$ + . . . + $A^n$ = \alert<4>{($A^1$ + . . . + $A\textsuperscript{n/2}$)} + ($A^1$ + . . . + $A\textsuperscript{n/2}$) $\times$ $A^\textsuperscript{n/2}$}
    \pause
    \par{\textbf{¡Subproblema!}}
    
    
\end{frame}

\begin{frame}{¿Estamos resolviendo el problema?}
    Importante: estamos haciendo menos operaciones de potencia. ¿Pero cuantas?
    
    \pause
    \vspace{2em}
    Escribamos un pseudocódigo y despues medimos la complejidad en términos de la función potencia.
    
\end{frame}

\begin{frame}{Pseudocódigo}
    
\begin{algorithmic}

\Function {suma\_geometrica}{A, n}
    \If {n == 1}
	\State \Return{A} \Comment{A = A\textsuperscript{1}}
    \EndIf

    \vspace{0.5em}
    \State $subsuma \gets suma\_geometrica(A, n/2)$ 
    \vspace{0.5em}
    \State $pot \gets potencia(A, n/2)$ \Comment{A\textsuperscript{n/2}}
    \vspace{0.5em}
    
    \State \Return {$subsuma + subsuma \times pot$} 
    \vspace{0.5em}
    
\EndFunction

\pause

¿Complejidad (en función de operaciones potencia)?: 
\pause
 \[ P(n) =  \begin{cases} 
    \mathcal{O}(1) & n = 1 \\
    P(\frac{n}{2}) + \mathcal{O}(1) & n > 1 \\ 
   \end{cases}
 \]


\end{algorithmic}

\end{frame}

\begin{frame}{El Teorema Maestro}
\[ P(n) =  \begin{cases} 
    \Theta(1) & n = 1 \\
    P(\frac{n}{2}) + \Theta(1) & n > 1 \\ 
   \end{cases}
 \]
 
 ¿Podemos usar el Teorema Maestro? Veamos sus casos:
 
 \pause
 
 \vspace{0.5em}
 
 \par{$P(n) = aT(n/b) + f(n)$}
 
 \vspace{0.5em}
 
 \begin{enumerate}
    \item $f(n) = $  $\mathcal{O}$(n \textsuperscript{$log_b{a} $ - \(\epsilon\)  })  para algun \(\epsilon\) > 0 $\implies$ $P(n) =$ $\Theta$(n \textsuperscript{$log_b{a} $ })
    \item $f(n) = $  $\Theta$(n \textsuperscript{$log_b{a} $ }) $\implies$ $P(n) =$ $\Theta$(n \textsuperscript{$log_b{a} $ } lg n)
    \item $f(n) = $  $\Omega$(n \textsuperscript{$log_b{a} $ + \(\epsilon\) }) para algun \(\epsilon\) > 0, $a f(n/b) \leq c f(n)$ para algún c > 1 y todo n $\geq$ $n_0$ 
    $\implies$ $P(n) =$ $\Theta(f(n))$
 \end{enumerate}

\pause
\par{b = 2, a = 1, $log_b{a} = 0$ $\implies$ Estamos en el segundo caso. }
\par{$P(n) = \Theta$(lg n)}   

\end{frame}

\begin{frame}{Ejercicio Adicional}
    \par{Supongamos que la multiplicación de matrices es $\mathcal{O}(1)$}
    \par{¿Como es la complejidad en función de operaciones elementales?}
    
    \pause
    \vspace{1em}
    ¿Como varía segun las implementaciones de potencia?
    
    \pause
    \vspace{1em}
    Para pensar:
    
    \begin{enun}
	Ejercicio 3 de la práctica:
	Encuentre un algoritmo para calcular a\textsuperscript{b} en $\mathcal{O}$(lg b) . Piense cómo reutilizar los resultados
	ya calculados
	
    \end{enun}
    
\end{frame}

\begin{frame}
    ¿Preguntas?
\end{frame}

\end{document}
