\documentclass[pdf]{beamer}
\newcommand{\quotes}[1]{``#1''}

\mode<presentation>
{

  \usetheme{Warsaw}
  %\usetheme{Boadilla}
  \useoutertheme{infolines}
  \usecolortheme[RGB={28,13,151}]{structure}
  \usetheme[height=7mm]{Rochester}
  % or ...

  \setbeamercovered{invisible}
  % or whatever (possibly just delete it)
  
}

\usepackage{ upgreek }
\usepackage{verbatim} 
\usepackage{listings}
\usepackage{tikz}
\usepackage{algpseudocode}

\usetikzlibrary{arrows}
\usetikzlibrary{shapes}
\tikzstyle{block}=[draw opacity=0.7,line width=1.4cm]

%\lstloadlanguages{C++}
%lstnewenvironment{code}
%	{%\lstset{	numbers=none, frame=lines, basicstyle=\small\ttfamily, }%
%	 \csname lst@SetFirstLabel\endcsname}
%	{\csname lst@SaveFirstLabel\endcsname}
%\lstset{% general command to set parameter(s)
%	language=C++, basicstyle=\footnotesize\ttfamily, keywordstyle=\slshape,
%	emph=[1]{tipo,usa}, emphstyle={[1]\sffamily\bfseries},
%	morekeywords={tint,forn,forsn},
%	basewidth={0.47em,0.40em},
%	columns=fixed, fontadjust, resetmargins, xrightmargin=5pt, xleftmargin=15pt,
%	flexiblecolumns=false, tabsize=2, breaklines,	breakatwhitespace=false, extendedchars=true,
%	numbers=left, numberstyle=\tiny, stepnumber=1, numbersep=9pt,
%	frame=l, framesep=3pt,
%} % CAMBIADO

\usepackage{multicol}

\usepackage[spanish]{babel}
% or whatever

\usepackage[utf8]{inputenc}
% or whatever

\usepackage{times}
\usepackage[T1]{fontenc}
% Or whatever. Note that the encoding and the font should match. If T1
% does not look nice, try deleting the line with the fontenc.

%\usepackage{enumitem} CAMBIADO
%\usepackage{booktabs} CAMBIADO

\newtheorem{enun}{Enunciado}
\newtheorem{idea}{Idea importante}


\title[Prueba de oposición] % (optional, use only with long paper titles)
{Prueba de oposición - Algoritmos 2016}

\author[Brian Bokser] % (optional, use only with lots of authors)
{Brian Bokser}
% - Give the names in the same order as the appear in the paper.
% - Use the \inst{?} command only if the authors have different
%   affiliation.
\institute[UBA-FCEN] % (optional, but mostly needed)
{
  Facultad de Ciencias Exactas y Naturales\\
  Universidad de Buenos Aires
}


% Ac� se puede insertar el logo de la UBA
% \pgfdeclareimage[height=0.5cm]{university-logo}{university-logo-filename}
% \logo{\pgfuseimage{university-logo}}


% Delete this, if you do not want the table of contents to pop up at
% the beginning of each subsection:
\AtBeginSubsection[]
{

  \begin{frame}<beamer>{Contenidos}
    \tableofcontents[currentsection,currentsubsection]
  \end{frame}
	
}


% If you wish to uncover everything in a step-wise fashion, uncomment
% the following command: 

%\beamerdefaultoverlayspecification{<+->}


\begin{document}
\pgfdeclarelayer{background}
\pgfsetlayers{background,main}
\begin{frame}
  \titlepage
\end{frame}

\begin{frame}{Contenidos}
  \tableofcontents
  % You might wish to add the option [pausesections]
\end{frame}



\section{Introducci\'on}

\begin{frame}{Introducci\'on}
    \begin{itemize}
	\item Materia :  \emph{Sistemas Operativos}
	\vspace{2em}
	\item Práctica : \emph{Tercera práctica: Sincronización}

    \end{itemize}

\end{frame}

\begin{frame}{Contexto}
    \par{Los alumnos ya tuvieron las dos teóricas de Sincronización }
    \vspace{2em}
	
    \par{Conocen las ideas clásicas, ahora tienen que ejercitarlas.}
    \vspace{2em}
	
\end{frame}

\section{Enunciado y motivación}

\begin{frame}{Enunciado}

\begin{enun}
    Suponga que se tienen N procesos $P_i$, cada uno de los cuales ejecuta un conjunto de sentencias $a_i$
y $b_i$ . Se los quiere sincronizar de manera tal que los $b_i$ se ejecuten \textbf{después de que se hayan ejecutado
todos los $a_i$}.

\end{enun}

\pause

\begin{algorithmic}

    \Function{P}{i} \Comment{El proceso i, sin sincronización}
        \State a(i)
        \vspace{1em}
        \State b(i)
        \vspace{1em}
        
    \EndFunction
\end{algorithmic}
    

\end{frame}

\begin{frame}{Motivación}
    \begin{itemize}
        \item Es conocido como Barrera o Rendezvous, y es fundamental.
        \vspace{1em}
        \item Aparece como subproblema de otros ejercicios.
        \vspace{1em}
        \item Es un buen ejemplo para entender los conceptos de sincronización. 
    \end{itemize}
\end{frame}

\section{Resolución}
\begin{frame}
    Primer idea: \pause Turnos! 
    \pause 
    \newline $a_1$ hasta $a_n$, luego $b_1$ hasta $b_n$.
    Aplicación directa del \quotes{problema de los turnos}
    
    \vspace{2em}
    
    \pause
    \begin{algorithmic}
    
        \Function{P}{i} \Comment{Idea de la implementación}
            \State sem\_a[i].wait()
            \vspace{1em}
            \State a(i)
            \vspace{1em}
            
            \State sem\_a[i+1].signal()
            \vspace{1em}
            
            \State sem\_b[i].wait()
            \vspace{1em}
            
            \State b(i)
            \vspace{1em}
            
            \State sem\_b[i+1].signal()
        
        \EndFunction
        
    \end{algorithmic}
    
    \pause
\end{frame}

\begin{frame}{Problemas}
    Problema: Queremos la mayor \textbf{Concurrencia} posible. 
    No queremos garantizar un orden de los $a_i$ ni de los $b_i$ 
    
    \vspace{2em}
    
    \pause
    
    Segunda idea: Contar cuantos ejecutaron a(i)
\end{frame}

\begin{frame}{Que vamos a necesitar}
    \begin{enumerate}
        \item contador = 0
        \item mutex\_contador = Mutex(1)
        \item sem\_barrera = Semaforo(0)
    \end{enumerate}
\end{frame}


\begin{frame}
    \huge{¿Preguntas?}
    \begin{center}
        \includegraphics[scale=0.4]{img/question-mark.jpg}
    \end{center}
    \large{¡Muchas gracias!}
\end{frame}

\end{document}
