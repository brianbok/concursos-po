\documentclass{beamer}


\usepackage[utf8]{inputenc}%esto permite meter tildes sin el coso
\usepackage[spanish]{babel}
\usepackage{listings}
\usepackage{algpseudocode}

\usepackage{amsmath}
\newcommand{\quotes}[1]{``#1''}
\newcommand{\bigo}[0]{$\mathcal{O}$}
%e\newcommand{\elevar}[2]{$ {#1}^{#2} $ }

% There are many different themes available for Beamer. A comprehensive
% list with examples is given here:
% http://deic.uab.es/~iblanes/beamer_gallery/index_by_theme.html
% You can uncomment the themes below if you would like to use a different
% one:
\usetheme{Madrid}


\newtheorem{idea-imp}{¡Idea Importante!}


\title{Concurso Area Teoría}

% A subtitle is optional and this may be deleted
%\subtitle{Optional Subtitle}


\author{Brian Bokser}

% - Give the names in the same order as the appear in the paper.
% - Use the \inst{?} command only if the authors have different
%   affiliation.

%\institute[UBA] % (optional, but mostly needed)

\date{}
% - Either use conference name or its abbreviation.
% - Not really informative to the audience, more for people (including
%   yourself) who are reading the slides online

%\subject{Theoretical Computer Science}
% This is only inserted into the PDF information catalog. Can be left
% out. 

% If you have a file called "university-logo-filename.xxx", where xxx
% is a graphic format that can be processed by latex or pdflatex,
% resp., then you can add a logo as follows:

% \pgfdeclareimage[height=0.5cm]{university-logo}{university-logo-filename}
% \logo{\pgfuseimage{university-logo}}

% Delete this, if you do not want the table of contents to pop up at
% the beginning of each subsection:
\AtBeginSubsection[]
{
  \begin{frame}<beamer>{Outline}
    \tableofcontents[currentsection,currentsubsection]
  \end{frame}
}

% Let's get started
\begin{document}

\maketitle

\begin{frame}{Contexto}
	\begin{itemize}
		\item Los alumnos están familiarizados con el lenguaje S
		\item Los alumnos tuvieron la teórica de codificación de programas y diagonalización
		\item El ejercicio podría ser un ejercicio de parcial
	\end{itemize}
\end{frame}


\begin{frame}{Enunciado}
	Un programa P en el lenguaje S se dice optimista si $\forall$ i = 1, ..., n
	
	\medskip 
	
	si $I_{i}$ es la instrucción IF V $\neq$ 0 GOTO L entonces L no aparece como etiqueta de ninguna instrucción $I_{j}$ con j $\leq$ i

  	\medskip

	Sea 
	$r(x) = \begin{cases}
       1 & $si el programa de numero x es optimista $\\
       0 & $si no $\\
  	\end{cases}$

  	\medskip

  	Demostrar que el predicado r(x) es primitivo recursivo. 


\end{frame}


\begin{frame}{Motivación}
	\begin{itemize}
		\item Entender la codificación de programas (programas como números)
		\item Practicar conceptos de recursividad primitiva
		\item Dividir en subproblemas
	\end{itemize}
\end{frame}


\begin{frame}{Metodología}
	\begin{idea-imp}
		Resolución \textbf{Top-Down}: Dividir el problema en \textbf{subproblemas}
	\end{idea-imp}

	\bigskip
	\pause

	Los subproblemas son definiciones del enunciado.
	
	\bigskip
	\pause

	Si el enunciado nos habla de instrucciones válidas, saber identificar una de estas puede ser un subproblema 

	\bigskip
	\pause
	Tratar de resolver cada subproblema \quotes{en un linea} 

\end{frame}

\begin{frame}{Codificación}
	\begin{center}
		\includegraphics[scale=0.35]{instrucciones.png}
	\end{center}
\end{frame}

\begin{frame}{Codificación}
	\begin{center}
		\includegraphics[scale=0.35]{programa.png}
	\end{center}
\end{frame}


\begin{frame}
	Primer problema: ¿Qué es r?
	
	\bigskip
	\pause

	\hfill Es optimista $\equiv$ No hay una instruccion que sea inválida 

	\medskip
	\pause


	r(x) $\equiv \alpha(\exists_{i \leq |x+1|}(esSaltoAtras(i, x+1))) $

	\bigskip
	\pause


	\hfill válida $\equiv$ Si tengo un if, no salta a una etiqueta anterior. 

	\medskip
	\pause


	esSaltoAtras(i, x) $ \equiv$ \\
	$esUnIf(i, x) \implies \exists_{j \leq i}(etiqueta(j, x) = etiquetaDeIf(x[i]) ) $

	\bigskip

	
\end{frame}
\begin{frame}
	
	\hfill Es un if si $a = \#(L) + 2$ en $<a, <b, c> >$ 

	\medskip
	\pause

	esUnIf(i, x) $ \equiv l(r(x)) > 2$

	\bigskip
	\pause

	\hfill La etiqueta es la primer parte de la tripla

	\medskip
	\pause

	etiqueta(j, x) $ \equiv l(x[j])$

	\bigskip
	\pause

	\hfill Etiqueta de un if es la segunda parte menos dos

	\medskip
	\pause

	etiquetaDeIf(x) $ \equiv l(r(x)) - 2$
\end{frame}

\begin{frame}{Conclusiones}
	r(x) es primitivo recursivo pues es composición de p.r.

	\bigskip

	El existencial \textbf{acotado} es p.r.

	\bigskip

	Los observadores de lista y pares son p.r.

	\bigskip
	\bigskip
	\bigskip

	No olvidarse de mencionar esto en la justificación

\end{frame}

\begin{frame}{¿Preguntas?}
	\begin{center}
		\includegraphics[scale=0.4]{homero-pelicula.jpg}
	\end{center}
\end{frame}


\end{document}


